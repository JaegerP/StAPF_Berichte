\documentclass[compress, aspectratio=169]{beamer}

%presentation layout

\mode<presentation>
{
  \usetheme{Berlin}
  % \usecolortheme{dove}
  \setbeamercolor{structure}{bg=black,fg=white}
  \setbeamercolor{normal text}{bg=black,fg=white}
  \setbeamercolor{titlepage}{bg=black,fg=white}
  \setbeamercolor{titlelike}{bg=black,fg=white}
  \setbeamercolor{palette primary}{bg=black}
  \setbeamercolor{palette secondary}{bg=black, fg=gray}
  \setbeamercolor{palette tertiary}{bg=black, fg=gray}
  \setbeamercolor{palette quarternary}{bg=black}
  \setbeamercovered{transparent}
  \useinnertheme{rectangles}
  %\usefonttheme{serif}
}

\setbeamertemplate{navigation symbols}{}

%loading packages
\usepackage[ngerman]{babel}
\usepackage[T1]{fontenc}
\usepackage[utf8]{inputenc}
\usepackage{graphicx}
\usepackage{amsmath}
\usepackage{framed}

% vorgeplaenkel
\title[StAPf-Bericht]{StAPF-Bericht}

\author{Ständiger Ausschuss aller Physikfachschaften}

\institute[Zusammenkunft aller Physikfachschaften]

\date{10. November 2016}

\subject{ZäPFchen-Einführung}

\begin{document}

\begin{frame}[plain]{}
  \titlepage
\end{frame}

\section{Zusammensetzung}

\begin{frame}{Aktuelle Zusammensetzung}
	\begin{itemize}
		\item \emph{Karola Schulz} (Uni Potsdam)
		\item \emph{Katharina Meixner} (Uni Frankfurt)
		\item Andre Kreuzburg (Uni Düsseldorf)
		\item Lukian Bottke (Uni Würzburg)
		\item Maria Schlungbaum (TU Berlin)
	\end{itemize}
\end{frame}

\section{Was bisher geschah...}

\begin{frame}{Was bisher geschah...}
	\begin{itemize}
		\item Resolutionen verschickt und veröffentlicht
			\begin{itemize}
				\item Gleichbehandlung aller Statusgruppen bei Quotenregelungen
				\item Studentische Beschäftigungsverhältnisse nach dem WissZeitVG
				\item Drittmittelforschung
				\item Zwei-Studiensysteme
				\item Exzellenzinitiative III
			\end{itemize}
	\end{itemize}
\end{frame}

\begin{frame}
	\begin{itemize}
		\item ZaPF-Bericht verschickt und veröffentlicht
		\item 9 Sitzungen seit der ZaPF am See
		\item Klausurtagung in Frankfurt zur Nachbereitung Konstanz
		\item Klausurtagung in Berlin zur Vorbereitung Dresden
	\end{itemize}
	\vspace{5mm}
	\begin{center}
		\Large DANKE an alle, die uns dabei unterstützt haben!
	\end{center}
\end{frame}

\section{Rückmeldungen}

\begin{frame}{Rückmeldungen auf die Gleichbehandlung aller Statusgruppen bei Quotenregelungen}
	\begin{itemize}
		\item SPD Bundestagsfraktion
		\item CDU/CSU Bundestagsfraktion
		\item AFD Landtagsfraktion Thüringen
		\item Abgeordnetenbüro Hamburg - Stadtentwicklung, Kultur und Gleichstellung
		\item Senatorin für Wissenschaft, Gesundheit und Verbraucherschutz, Bremen
	\end{itemize}
\end{frame}

\section{VG Wort}

\begin{frame}{Beschluss zum VG Wort}
	\begin{itemize}
		\item Kommunikation mit den anderen BuFaTas
		\item Beschluss zum Brief der FaTaMa zum VG Wort
		\item[$\rightarrow$] Alles weitere im AK VG Wort
	\end{itemize}
\end{frame}

\section{Akkreditierung}

\begin{frame}{Akkreditierungspool}
	\begin{itemize}
		\item[$\rightarrow$] Auslaufende Mandate:
			\begin{itemize}
				\item Markus Gleich
				\item Margret Heinze
				\item Björn Guth
				\item Thomas Kirchner
				\item Katharina Meixner
				\item Jannis Andrija Schnitzer
			\end{itemize}
		\item PVT
			\begin{itemize}
				\item Letztes Treffen 06./07.08.2016 in Kiel\\
					Es waren zwei ZaPFika dort.
				\item Nächstes Treffen steht noch nicht fest.
			\end{itemize}
	\end{itemize}
\end{frame}

\begin{frame}
	\begin{framed}
		\begin{center}
			{\Huge \textbf{Wichtig}}\\
			\vspace{0.5cm}
			{\Large Aktuelle Anmeldeformulare für den Pool ausfüllen und in digitaler Form an die Verwaltung senden}
		\end{center}
	\end{framed}
\end{frame}

\section{MeTaFa}

\begin{frame}{MeTaFa}
	\begin{itemize}
		\item Treffen in Oldenburg (23. bis 25. September 2016)
			\begin{itemize}
				\item ZaPF vertreten durch Christian (Uni Oldenburg)
			\end{itemize}
		\item Nächstes Treffen steht noch nicht endgültig fest, Dresden und Saarbrücken sind im Gespräch für einen Termin im Frühjahr 2017
		\item Diskussion u.a. über ein Antwortschreiben der HRK zum Thema Semesterzeiten, Attestpflicht, bundesweites Semesterticket, ...
	\end{itemize}
\end{frame}

\section{Kommende ZaPFen}
\begin{frame}{Kommende ZaPFen}
	\begin{itemize}
		\item Sommersemester 2017 in Berlin
		\item Wintersemester 2017 in Siegen
		\item Sommersemester 2018 in Heidelberg
		\item zu Vergeben:
			\begin{itemize}
				\item Wintersemester 2018
				\item ...
			\end{itemize}
		\item bisherige Bewerbungen:
			\begin{itemize}
				\item Für Wintersemester 2018: Würzburg
			\end{itemize}
	\end{itemize}
\end{frame}

\begin{frame}[plain]
	\begin{center}
		\Huge Habt ihr Fragen an uns?
	\end{center}
\end{frame}

\section{TOPF}
\begin{frame}{TOPF}
	\begin{itemize}
		\item[] Deckel\footnote{Ihr wisst schon, die mit den kaputten Laptops}:
			\begin{itemize}
				\item \emph{Fabian Freyer} (TU Berlin)
				\item Robert Löffler (Uni Konstanz)
			\end{itemize}
	\end{itemize}
\end{frame}

\begin{frame}{TOPF}
	\begin{itemize}
		\item[] Der TOPF hat einen Server für Mailinglisten eingerichtet. StAPF und TOPF sind jetzt erreichbar unter
			\begin{itemize}
				\item stapf@zapf.in
				\item topf@zapf.in
			\end{itemize}
		\item[] Einen ausführlichen Bericht über die Tätigkeiten des TOPFs wird im eigenen AK vorgestellt
	\end{itemize}
\end{frame}

\section{KommGrem}

\begin{frame}{KommGrem}
	\begin{itemize}
		\item[] ZaPF:
			\begin{itemize}
				\item \emph{Thomas Rudzki} (Uni Heidelberg)
				\item \emph{Zafer El-Mokdad} (Alter Sack)
			\end{itemize}
		\item[] jDPG:
			\begin{itemize}
				\item Eric Abraham (Jena)
				\item Hejo Kerl (Zürich)
			\end{itemize}
	\end{itemize}
	\vspace{0.5cm}
	\textbf{Sprecher}: Eric Abraham
\end{frame}

\section{LEUTE zur SACHE}

\begin{frame}{LEUTE zur SACHE\footnote{\textbf{L}ieblings \textbf{E}ngagierte in \textbf{U}ngewählter \textbf{T}askforc\textbf{E} zur \textbf{S}ach\textbf{A}rbeit am \textbf{CHE}}}
	\begin{itemize}
		\item \emph{Valentin Wohlfarth, Margret Heinze, Christian Hoffmann, Tim Luis Borck}
		\item 3 AKe vorbereitet:
			\begin{itemize}
				\item Diskussion zum CHE und Rankings allgemein
				\item CHE Info-Workshop
				\item CHE AK
			\end{itemize}
	\end{itemize}
\end{frame}

%\begin{frame}{KommGrem}
%		Bachelor-Master-Umfrage	- Was IHR jetzt damit machen könnt
%	\begin{itemize}
%		\item Korrelation Bachelor zu Master
%		\item Beispiele
%			\begin{itemize}
%				\item Studiengagnswechsel
%				\item Veränderung Studienwahlindikatoren
%				\item Übungsbetrieb
%				\item Rückblick Bachelorarbeit
%				\item ...
%			\end{itemize}
%		\item Vorschlag: Arbeitskreise zu diesen Themen?!
%		\item bisher haben nur 4 Fachschaften die Ergebnisse angefragt!
%	\end{itemize}
%\end{frame}

%\begin{frame}{KommGrem}
%\begin{itemize}
%\item Bericht von der Praktikumsleitertagung
%\item Ziele-Umfrage:\\
%Wiki $\rightarrow$ ZaPF Konstanz $\rightarrow$ Arbeitskreise $\rightarrow$ AK Praktika
%\end{itemize}
%\end{frame}

%\begin{frame}{KommGrem}
%\begin{itemize}
%\item CHE Positionspapier
%\item Antwort des CHE war positiv (offen für Kooperation)
%\item Ausblick
%\begin{itemize}
%\item weitere Kommunikation mit CHE
%\item Teilnahme an KFP
%\item Auswertung Bachelor-Master-Umfrage
%\end{itemize}
%\end{itemize}
%\end{frame}

\section{ZaPF e.V.}

\begin{frame}{ZaPF e.V.}
	\begin{itemize}
		\item[] aktueller Vorstand:
			\begin{itemize}
				\item Florian Marx (Uni Frankfurt) (Vorsitzender)
				\item Patrick Haiber (Uni Konstanz) (Kassenwart)
				\item Tobias Löffler (Uni Düsseldorf)
				\item Christoph Steinacker (TU Dresden)
				\item Valentin Wolfarth (HU Berlin) 
				\item Jan Luca Naumann (FU Berlin)
				\item Thomas Rudzki (Uni Heidelberg)
			\end{itemize}
	\end{itemize}
\end{frame} 
\end{document}
