\documentclass[compress, aspectratio=169]{beamer}

%presentation layout

\mode<presentation>
{
  \usetheme{Berlin}
  % \usecolortheme{dove}
  \setbeamercolor{structure}{bg=black,fg=white}
  \setbeamercolor{normal text}{bg=black,fg=white}
  \setbeamercolor{titlepage}{bg=black,fg=white}
  \setbeamercolor{titlelike}{bg=black,fg=white}
  \setbeamercolor{palette primary}{bg=black}
  \setbeamercolor{palette secondary}{bg=black, fg=gray}
  \setbeamercolor{palette tertiary}{bg=black, fg=gray}
  \setbeamercolor{palette quarternary}{bg=black}
  \setbeamercovered{transparent}
  \useinnertheme{rectangles}
  %\usefonttheme{serif}
}

\setbeamertemplate{navigation symbols}{}

%loading packages
\usepackage[ngerman]{babel}
\usepackage[T1]{fontenc}
\usepackage[utf8]{inputenc}
\usepackage{graphicx}
\usepackage{amsmath}
\usepackage{framed}

% vorgeplaenkel
\title[StAPf-Bericht]{StAPF-Bericht}

\author{Ständiger Ausschuss aller Physikfachschaften}

\institute[Zusammenkunft aller Physikfachschaften]

\date{30. Mai 2018}

\subject{Bericht des StAPF}

\begin{document}

\begin{frame}[plain]{}
  \titlepage
\end{frame}

\section{Zusammensetzung}

\begin{frame}{Aktuelle Zusammensetzung}
  \begin{itemize}
  \item \emph{Svenja Bramlage (Uni Bonn)}
  \item \emph{Niklas Donocik (TU Braunschweig)}
  \item \emph{Jennifer Hartfiel (FU Berlin)}
  \item Ann-Kathrin Klein (Uni Tübingen)
  \item Marcus Mikorski (Alumni)
  \end{itemize}
\end{frame}

\section{Was bisher geschah...}

\begin{frame}{Was bisher geschah...}
  \begin{itemize}
  \item Resolutionen verschickt und veröffentlicht
    \begin{itemize}
        \item Positionspapier \& Reso zur neuen Musterrechtsverordnung zur Akkreditierung
        \item Resolution, die die Anrechnung von Berufspraktika befürwortet
        \item Reso, die von der Exzellenintiative fordert, Lehre und Forschung gleichmäßig zu beachten
        \item Positionspapier zur Förderung der Wissenschaftskommunikation im Studium
        \item Positionspapier zur Rolle der Wissenschaftskommunikation im Studium        
        \item Resolution zur Beibehaltung des politischen Mandats der baden-würtembergischen Fachschaften
        \item Resolution gegen die Einführung von Studiengebühren in NRW
        \item Positionspapier, das ICAN zum Friedensnobelpreis gratuliert
        \item Resolution zum Umgang mit Nullergebnissen
        \item Resolution, die Arbeitsunfähigkeit der Prüfungsunfähigkeit gleichstellen will
        \item Resolution gegen Zwangsexmatrikulierung
    \end{itemize}
  \end{itemize}
\end{frame}

\begin{frame}
  \begin{itemize}
    \item ZaPF-Bericht verschickt und veröffentlicht
    \item 7 Sitzungen seit der ZaPF in Berlin
    \item Klausurtagung vom 17.11. bis 19.11.17 in Karlsruhe zur Nachbereitung von Siegen
    \item Klausurtagung vom 25.05. bis 27.05.18 in Würzburg zur Vorbereitung von Heidelberg
    \end{itemize}
    \vspace{5mm}
    \begin{center}
      \Large DANKE an alle, die uns dabei unterstützt haben!
    \end{center}
\end{frame}

\section{Rückmeldungen}

\begin{frame}{BMBF}
  \begin{itemize}
     \item Es gab eine Anfrage und zwei Nachfragen an das BMBF
     \item Antwort: Nicht genügend Mittel in Förderrunde 2017/2018
     \item Die nächsten ZaPFen sind nicht in Gefahr
  \end{itemize}
\end{frame}


\section{Akkreditierung}

%unvollständig
\begin{frame}{Akkreditierungspool}
  \begin{itemize}
    \item[$\rightarrow$] Auslaufende Mandate:
      \begin{itemize}
          \item [NAME]
      \end{itemize}
    \item PVT
      \begin{itemize}
          \item [INFOS]
      \end{itemize}
    \end{itemize}
\end{frame}

\begin{frame}
  \begin{framed}
    \begin{center}
      {\Huge \textbf{Wichtig}}\\
      \vspace{0.5cm}
      {\Large Aktuelle Anmeldeformulare für den Pool ausfüllen und in digitaler Form an die Verwaltung senden}
      \end{center}
      \end{framed}
\end{frame}

% \section{MeTaFa}

% \begin{frame}{MeTaFa}
%   \begin{itemize}
%   \item Treffen in Dresden (22. bis 24. September 2017 )
%     \begin{itemize}
%       \item Die ZaPF war dieses Mal nicht vertreten
%     \end{itemize}
%   \end{itemize}
% \end{frame}

\section{Kommende ZaPFen}
\begin{frame}{Kommende ZaPFen}
  \begin{itemize}
    \item Wintersemester 2018 in Würzburg
    \item Sommersemester 2019 in Bonn
    \item Sommersemester 2020 in Rostock
    \item zu Vergeben:
      \begin{itemize}
      \item Wintersemester 19
      \end{itemize}
    \item bisherige Bewerbungen:
      \begin{itemize}
      \item Hier könnte deine Uni stehen!
      \end{itemize}
    \end{itemize}
\end{frame}

\begin{frame}[plain]
  \begin{center}
    \Huge Habt ihr Fragen an uns?
    \end{center}
\end{frame}

\section{KommGrem}

\begin{frame}{KommGrem}
  \begin{itemize}
  \item[] ZaPF:
    \begin{itemize}
    \item \emph{Sonja Gehring (Uni Bonn)}
    \item Niklas Brandt (Uni Oldenburg)
    \end{itemize}
  \item[] jDPG:
    \begin{itemize}
    \item Eric Abraham (Jena)
    \item Merten Dahlkemper (Göttingen)
    \end{itemize}
  \end{itemize}
  \vspace{0.5cm}
%  \textbf{Sprecher}: 
\end{frame}


\section{ZaPF e.V.}

\begin{frame}{ZaPF e.V.}
Hier könnte ein Bericht des e.V. stehen.
\end{frame} 
\end{document}
%%% Local Variables:
%%% mode: latex
%%% TeX-master: t
%%% End:
